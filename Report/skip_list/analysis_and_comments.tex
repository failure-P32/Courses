\section{Analysis and Comments}
Because skip list is based on a randomized algorithm, all the complexities analyzed below is the mathematical expectation. The worst-case complexity might be worse, but only with a very small probability.
\subsection{Time Complexity}
In this section, we will prove that the expectation of time complexity of search, insertion and deletion is $O(\log{N})$.
\subsubsection{The expected $Max\_level$}
Let $p$ denotes the probability of increasing the level in the random\_level algorithm \ref{alg:rand} (In our implementation, $p=\frac 1 2$). For an arbitary node, let $k$ denotes the level of it, then $P(k=i)=p^i(1-p)$(i.e. Geometric distribution). Then, the expected number of elements in the $k_{th}$ level (denoted $N_k$) is 
\[
    E(N_k) = \left\{
        \begin{aligned}
            &N,\ k=0\\
            &N(1-\sum_{i=0}^{k-1}p^i(1-p))=N(1-\frac{(1-p)(1-p^k)}{1-p})=Np^k,\ k\geq 1
        \end{aligned}
    \right.
\]
To make the search efficient, we want the highest level $M-1$($M$ is the total number of levels) has an expected $O(1)=a$ number of elements, i.e.
\begin{align*}
    E(N_{M-1})=Np^{M-1}&=a\\
    \log_p{N}+M-1&=\log_p{a}\ \text{(Logarithm on both sides)}\\
    \therefore M=-O(\log_p{N})&=O(\log_\frac{1}{p}N)
\end{align*}
Therefore, the $Max\_level$ of the skip list can be set to $\log_\frac{1}{p}N$ to obtain maximum expected efficiency. In computer, the range of integer is $[0,2^{32})$, so $Max\_level$ is often set to be 32 in practical.
\subsubsection{The expected search cost}
We use a reverse analysis method to analyze the cost. Suppose we have found the node in the $k_{th}$ level, we analyze the search path backwards, going up and left to the head $H$.\par
At an arbitary node $x$ in the $i_{th}$ level in the backward-path, there are 2 situations:
\begin{enumerate}
    \item The level of node $x$ is exactly $i$, then the next step is to go left.\label{enu:case1}
    \item The level of node $x$ is larger than $i$, then the next step is to continue to climb up.\label{enu:case2}
\end{enumerate}
Let's take the case of finding 7 in section 2 as an example. The dashed arrows denote the path backwards. The red 6 and 4 nodes satisfy 2 situations above respectively:
\begin{figure}[H]
    \centering
    \begin{tikzpicture}[shorten >=1pt,node distance=1.5cm,on grid,
        place/.style={circle,draw=blue!50,fill=blue!20,text=white,minimum size=10mm},
        sit/.style={circle,draw=red!50,fill=red!20,text=white,minimum size=10mm},
        every state/.style={fill=none,draw=black,text=black,minimum size=10mm}]
        
        \node[place] (H1)  at (0,3) {$H$};
        \node[place] (n11) at (6,3) {$4$};
        \node[state] (n12) at (12,3) {$8$};
        \node[state] (T1) at (13.5,3) {$T$};

        \node[state] (H2)  at (0,1.5) {$H$};
        \node[state] (n21) at (3,1.5) {$2$};
        \node[sit] (n22) at (6,1.5) {$4$};
        \node[sit] (n23) at (9,1.5) {$6$};
        \node[state] (n24) at (12,1.5) {$8$};
        \node[state] (T2) at (13.5,1.5) {$T$};

        \node[state] (H3)  at (0,0) {$H$};
        \node[state] (n31) at (1.5,0) {$1$};
        \node[state] (n32) at (3,0) {$2$};
        \node[state] (n33) at (4.5,0) {$3$};
        \node[state] (n34) at (6,0) {$4$};
        \node[state] (n35) at (7.5,0) {$5$};
        \node[place] (n36) at (9,0) {$6$};
        \node[place] (n37) at (10.5,0) {$7$};
        \node[state] (n38) at (12,0) {$8$};
        \node[state] (T3)  at (13.5,0) {$T$};

        \path[->] 
                  (H1)  edge [blue!50] node []{} (n11)
                  (n11) edge [dashed, bend right] node []{} (H1)
                  (n11) edge [] node []{} (n12)
                  (n12) edge [] node []{} (T1)

                  (H2)  edge [] node []{} (n21)
                  (n21) edge [] node []{} (n22)
                  (n22) edge [blue!50] node []{} (n23)
                  (n23) edge [dashed, bend right] node []{} (n22)
                  (n23) edge [] node []{} (n24)
                  (n24) edge [] node []{} (T2)
        
                  (H3)  edge [] node []{} (n31)
                  (n31) edge [] node []{} (n32)
                  (n32) edge [] node []{} (n33)
                  (n33) edge [] node []{} (n34)
                  (n34) edge [] node []{} (n35)
                  (n35) edge [] node []{} (n36)
                  (n36) edge [blue!50] node []{} (n37)
                  (n37) edge [dashed, bend right] node []{} (n36)
                  (n37) edge [] node []{} (n38)
                  (n38) edge [] node []{} (T3)
                  
                  (H1)  edge [] node []{} (H2)
                  (H2)  edge [] node []{} (H3)
                  
                  (n21)  edge [] node []{} (n32)
                  
                  (n11)  edge [blue!50] node []{} (n22)
                  (n22)  edge [dashed, bend right] node []{} (n11)
                  (n22)  edge [] node []{} (n34)
                  
                  (n23)  edge [blue!50] node []{} (n36)
                  (n36)  edge [dashed, bend right] node []{} (n23)

                  (n12)  edge [] node []{} (n24)
                  (n24)  edge [] node []{} (n38)
                  
                  (T1)  edge [] node []{} (T2)
                  (T2)  edge [] node []{} (T3);
    \end{tikzpicture}
    \caption{The search path backwards from 7}\label{fig:back}
\end{figure}
The red node 6 is in level 1, equals to the maximum level of itself, so satisfies case \ref{enu:case1}; while the red node 4 is in level 1 but the maximum level of itself is 2, which satisfies \mbox{case \ref{enu:case2}.}\par
Because the level follows geometric distribution, in an arbitary node $x$, it has a probability of $1-p$ to satisfy case \ref{enu:case1} while $p$ to satisfy case \ref{enu:case2}. Let $C(k)$ denotes the expected cost of search path that climbs up $k$ level, then
\begin{align*}
    C(0) &= 0\\
    C(k) &= (1-p)(1+\text{cost in situation \ref{enu:case2}})+p(1+\text{cost in situation \ref{enu:case1}})\\
    &= (1-p)(C(k)+1) + p(C(k-1)+1)\\
    \therefore C(k) &= \frac{1}{p} + C(k-1)\\
    \therefore C(k) &= \frac{k}{p}
\end{align*}
Because $k\leq Max\_level = O(\log_{\frac 1p}{N})$, thus $C(k)=O(\log{N})$. The time complexity of search is proportional to the cost of the search path, thus we finish the proof that the expected time of search is $O(\log{N})$.
\subsubsection{The expected insertion/deletion cost}
The insertion/deletion takes 2 steps:
\begin{enumerate}
    \item Find the predecessor
    \item Insert/Delete
\end{enumerate}
Step 1 costs $O(\log{N})$ while step 2 costs $O(k)\leq Max\_level = O(\log{N})$ where $k$ is the level of the node to insert/delete, thus the expected time of insertion/deletion is also $O(\log{N})$.

\subsection{Space complexity}
The level $L$ follows geometric distribution, so the expectation of $L$ is
\[
    E(L) = \sum_{k=0}^\infty kp^k(1-p) = \frac{1}{1-p}
\]
For each node, it costs $O(L)$ space to store the successors array \texttt{next\_nodes}. Therefore, the space complexity of a skip list is $O(NL)=O(\frac{N}{1-p})=O(N)$. The smaller $p$ is, the less space the skip list costs.\par
Here, we assume that the maximum level is infinity. If max level is finite (denoted $M$), we have
\[
    P(L=k)=\left\{
        \begin{aligned}
            &p^k(1-p),\ k<M-1\\
            &1-\sum_{k=0}^{M-2}p^k(1-p)=p^{M-1},\ k=M-1
        \end{aligned}
        \right.
\]
When $M=O(\log{N})$ is large, $(M-1)p^{M-1}\to0$, the bias caused by assuming infinity level can be ignored. Thus, for skip list with finite level, the space complexity is still $O(N)$. 

\subsection{Further thinking}
As an emerging data structure, skip list exudes its unique brilliance with its high efficiency and simplicity. Compared to balanced search trees, skip list is not inferior to them.\par
For a long time, in order to optimize the efficiency of searching, a lot of sophisticated tree structures are designed. Despite their efficiency, their large programming complexity is unacceptable. Skip list not only describes the structure itself, but also ``skips'' the stereotype of the ``tree'' structure. As a data structure, skip list has been widely used to describe an ordered set instead of using trees in more and more areas. As a way of thinking, ``skipping the stereotype'' leads mankind to making progress one after another!